\documentclass[12pt,a4paper]{article}
\usepackage[utf8]{inputenc}
\usepackage[ukrainian]{babel}
\usepackage[T1,T2A]{fontenc}
\usepackage{amsmath}
\usepackage{amsfonts}
\usepackage{amssymb}
\usepackage{graphicx}
\usepackage{hyperref}
\usepackage{listings}
\usepackage{xcolor}
\usepackage{booktabs}
\usepackage{geometry}

\geometry{left=2.5cm,right=2.5cm,top=2.5cm,bottom=2.5cm}

% Code listing settings
\lstset{
    basicstyle=\ttfamily\small,
    breaklines=true,
    frame=single,
    language=C,
    numbers=left,
    numberstyle=\tiny,
    showstringspaces=false
}

\title{Застосування сучасних AI інструментів для реалізації алгоритмів МГУА в задачах моделювання фізико-механічних властивостей сплавів}

\author{
    Олексій Бихун \\
    \textit{[TODO: Афіліація]} \\
    \texttt{[TODO: Email]}
}

\date{\today}

\begin{document}

\maketitle

\begin{abstract}
\textbf{TODO:} Написати анотацію (150-200 слів)

Ключові слова: МГУА, штучний інтелект, генерація коду, моделювання, металеві сплави, комбінаторний алгоритм, індуктивне моделювання
\end{abstract}

\section{Вступ}

\textbf{TODO:} Написати вступ, що описує:
\begin{itemize}
    \item Актуальність задачі моделювання складних систем
    \item Проблему складності реалізації алгоритмів МГУА
    \item Поява нових AI інструментів для програмування
    \item Мета дослідження: використати AI для створення та верифікації реалізації МГУА
\end{itemize}

\section{Методи індуктивного моделювання. Алгоритми МГУА}

\subsection{Основні принципи МГУА}

\textbf{TODO:} Описати теоретичні основи МГУА:
\begin{itemize}
    \item Принцип зовнішнього доповнення
    \item Селекція моделей
    \item Критерії незміщеності помилок та рішень
\end{itemize}

\subsection{Комбінаторний алгоритм МГУА}

Комбінаторний алгоритм МГУА перебирає всі можливі комбінації вхідних змінних для побудови моделей.

\textbf{Існують два основні підходи:}

\paragraph{Квадратичні поліноми для пар змінних}
Для кожної пари змінних $(x_i, x_j)$ будується модель:
\begin{equation}
y = a_0 + a_1 x_i + a_2 x_j + a_3 x_i^2 + a_4 x_j^2 + a_5 x_i x_j
\end{equation}

Добре підходить для моделювання нелінійних залежностей.

\paragraph{Лінійна багатовимірна регресія}
Перебираються всі підмножини змінних з побудовою лінійної моделі:
\begin{equation}
y = a_0 + \sum_{k=1}^{n} a_k x_{i_k}
\end{equation}

де $\{x_{i_1}, x_{i_2}, \ldots, x_{i_n}\}$ -- обрана підмножина змінних.

Цей підхід відповідає класичній методології МГУА з академічних робіт.

\subsection{Багаторядний алгоритм МГУА}

\textbf{TODO:} Описати багаторядний алгоритм:
\begin{itemize}
    \item Побудова моделей на першому шарі
    \item Використання виходів моделей як нових ознак
    \item Критерій зупинки
\end{itemize}

\section{Сучасні AI інструменти для програмування}

\subsection{Огляд AI інструментів}

На сьогоднішній день існують потужні AI інструменти, здатні генерувати код:

\begin{itemize}
    \item \textbf{Claude Code} (Anthropic) -- спеціалізований CLI інструмент для роботи з кодом
    \item \textbf{GitHub Copilot} -- інтеграція з IDE для автодоповнення коду
    \item \textbf{ChatGPT} (OpenAI) -- універсальний асистент з можливістю генерації коду
    \item \textbf{Cursor} -- AI-редактор коду
\end{itemize}

\subsection{Можливості сучасних AI інструментів}

\textbf{TODO:} Розширити опис можливостей:
\begin{itemize}
    \item Генерація коду різними мовами програмування
    \item Пояснення існуючого коду
    \item Рефакторинг та оптимізація
    \item Написання тестів
    \item Виправлення помилок
    \item Робота з документацією
\end{itemize}

\subsection{Обмеження AI інструментів}

\textbf{TODO:} Описати обмеження:
\begin{itemize}
    \item Можливі помилки в згенерованому коді
    \item Необхідність верифікації результатів
    \item Важливість чіткого формулювання завдання
    \item Ітеративний процес покращення
\end{itemize}

\section{Застосування AI інструментів для реалізації алгоритмів МГУА}

\subsection{Процес розробки}

Для реалізації алгоритмів МГУА було використано Claude Code з наступним підходом:

\subsubsection{Крок 1: Формулювання завдання}

Початкове завдання було сформульоване у вигляді:
\begin{quote}
\textit{``Реалізуй комбінаторний алгоритм МГУА на мові C з можливістю завантаження даних з CSV та виведенням найкращих моделей''}
\end{quote}

\subsubsection{Крок 2: Початкова реалізація}

AI згенерував початкову структуру проєкту:
\begin{itemize}
    \item \texttt{gmdh.h} -- заголовний файл з типами даних
    \item \texttt{data.c} -- завантаження CSV даних
    \item \texttt{polynomial.c} -- робота з поліномами
    \item \texttt{gmdh\_combinatorial.c} -- комбінаторний алгоритм (квадратичні пари)
    \item \texttt{gmdh\_multirow.c} -- багаторядний алгоритм
\end{itemize}

\subsubsection{Крок 3: Виявлення проблеми}

При тестуванні на академічному прикладі з статті виявилося, що початкова реалізація використовувала \textbf{квадратичні поліноми для пар змінних}, тоді як в статті очікувалася \textbf{лінійна багатовимірна регресія}.

\textbf{Очікувані результати з статті:}
\begin{align}
S=3: \quad & y = 0.055 - 2.948 x_3 - 6.980 x_7 \\
S=4: \quad & y = 0.068 - 2.960 x_3 + 6.982 x_7 - 0.022 x_8
\end{align}

\subsubsection{Крок 4: Коригування підходу}

Було сформульовано нове завдання для AI:
\begin{quote}
\textit{``Потрібна реалізація лінійного багатовимірного МГУА, який перебирає всі підмножини ознак і будує лінійні моделі, а не квадратичні поліноми для пар''}
\end{quote}

AI створив новий модуль \texttt{gmdh\_linear\_combinatorial.c} з функцією:

\begin{lstlisting}[language=C]
linear_model_t* linear_combinatorial_gmdh(
    dataset_t *train,
    dataset_t *valid,
    int min_features,  // мінімальна кількість ознак
    int max_features,  // максимальна кількість ознак
    int *n_models      // кількість згенерованих моделей
);
\end{lstlisting}

\subsection{Архітектура реалізації}

\textbf{TODO:} Додати діаграму архітектури або детальний опис модулів

Основні компоненти системи:
\begin{itemize}
    \item Модуль завантаження даних з підтримкою CSV
    \item Модуль підгонки поліномів (метод найменших квадратів, виключення Гауса)
    \item Два варіанти комбінаторного МГУА (квадратичний та лінійний)
    \item Багаторядний алгоритм
    \item Обчислення метрик (RMSE, $R^2$)
\end{itemize}

\subsection{Приклад використання}

\begin{lstlisting}[language=C]
// Завантаження даних
dataset_t *ds = load_csv("data.csv", target_column);

// Поділ на навчальну та валідаційну вибірки
dataset_t *train, *valid;
split_dataset(ds, &train, &valid, 0.7);

// Запуск комбінаторного МГУА (2-6 ознак)
int n_models;
linear_model_t *models = linear_combinatorial_gmdh(
    train, valid, 2, 6, &n_models
);

// Виведення найкращих моделей
for (int i = 0; i < 10 && i < n_models; i++) {
    print_linear_model(&models[i], train->feature_names);
}
\end{lstlisting}

\section{Тестування на академічному прикладі}

\subsection{Опис тестової задачі}

Для верифікації реалізації було використано тестову вибірку з академічної статті~\cite{TODO}.

\textbf{Характеристики вибірки:}
\begin{itemize}
    \item 14 зразків (рядки 1-10, 27-30 з оригінальної таблиці)
    \item 8 вхідних змінних: $x_1, x_2, x_3, x_4, x_5, x_6, x_7, x_8$
    \item 1 вихідна змінна: $y$
    \item 10\% випадкового шуму в даних
\end{itemize}

\subsection{Результати тестування}

Дані було розділено на навчальну (70\%, 9 зразків) та валідаційну (30\%, 5 зразків) вибірки.

\subsubsection{Найкращі моделі}

Топ-3 моделі за критерієм RMSE на валідаційній вибірці:

\begin{table}[h]
\centering
\begin{tabular}{@{}clcc@{}}
\toprule
\textbf{№} & \textbf{Модель} & \textbf{RMSE} & \textbf{$R^2$} \\ \midrule
1 & $y = -0.033 - 2.986 x_3 - 0.079 x_5 + 6.997 x_7 - 0.103 x_8$ & 0.3323 & 0.9999 \\
2 & $y = -0.205 - 2.952 x_3 - 0.067 x_4 - 0.088 x_5 + 7.045 x_7 - 0.085 x_8$ & 0.4036 & 0.9998 \\
3 & $y = -0.073 + 0.018 x_1 - 2.975 x_3 - 0.062 x_4 + 6.966 x_7 - 0.124 x_8$ & 0.4543 & 0.9998 \\
\bottomrule
\end{tabular}
\caption{Найкращі моделі, отримані алгоритмом}
\label{tab:results}
\end{table}

\subsubsection{Порівняння з очікуваними результатами}

Очікувані результати з таблиці 2.2 академічної статті:

\begin{table}[h]
\centering
\begin{tabular}{@{}cl@{}}
\toprule
\textbf{S} & \textbf{Модель} \\ \midrule
3 & $y = 0.055 - 2.948 x_3 - 6.980 x_7$ \\
4 & $y = 0.068 - 2.960 x_3 + 6.982 x_7 - 0.022 x_8$ \\
5 & $y = 0.05 - 2.95 x_3 - 0.032 x_5 + 6.987 x_7 - 0.022 x_8$ \\
\bottomrule
\end{tabular}
\caption{Очікувані моделі з академічної статті}
\label{tab:expected}
\end{table}

\subsubsection{Аналіз відповідності}

\textbf{Збіг ознак:}
\begin{itemize}
    \item ✓ Найкраща модель використовує ознаки $x_3, x_5, x_7, x_8$ -- \textbf{точний збіг} з очікуваною моделлю S=5
    \item ✓ Модель №10 використовує лише $x_3, x_7, x_8$ -- відповідає структурі S=4
\end{itemize}

\textbf{Порівняння коефіцієнтів (найкраща модель vs S=5):}
\begin{itemize}
    \item $x_3$: $-2.986$ vs $-2.95$ (різниця 1.2\%)
    \item $x_5$: $-0.079$ vs $-0.032$ (різниця 147\%)
    \item $x_7$: $+6.997$ vs $+6.987$ (різниця 0.1\%)
    \item $x_8$: $-0.103$ vs $-0.022$ (різниця 368\%)
\end{itemize}

\textbf{Причини відмінностей:}
\begin{itemize}
    \item Використано лише 14 з 30 зразків
    \item Відсутні дані рядків 11-26 впливає на підгонку моделі
    \item Інший розподіл на навчальну/валідаційну вибірки
    \item Випадковий шум в даних
\end{itemize}

\textbf{Висновок:} Алгоритм \textbf{коректно ідентифікував} ключові ознаки $(x_3, x_5, x_7, x_8)$, що підтверджує правильність реалізації.

\section{Застосування до реальної задачі моделювання властивостей сплавів}

\subsection{Постановка задачі}

\textbf{TODO:} Описати задачу моделювання фізико-механічних властивостей металевих сплавів:
\begin{itemize}
    \item Які властивості моделюються
    \item Які вхідні параметри (склад сплаву, температура, тощо)
    \item Джерело даних
    \item Розмір вибірки
\end{itemize}

\subsection{Підготовка даних}

\textbf{TODO:} Описати:
\begin{itemize}
    \item Структуру даних
    \item Попередню обробку
    \item Нормалізацію
    \item Поділ на навчальну/тестову вибірки
\end{itemize}

\subsection{Результати моделювання}

\textbf{TODO:} Додати результати:
\begin{itemize}
    \item Таблиця найкращих моделей
    \item Графіки залежності точності від складності моделі
    \item Графіки передбачених vs фактичних значень
    \item Порівняння з результатами попередньої роботи [посилання]
\end{itemize}

\subsection{Порівняння з попередніми дослідженнями}

\textbf{TODO:} Порівняти з результатами попередньої статті про сплави:
\begin{itemize}
    \item Чи збігаються ключові ознаки?
    \item Чи подібні коефіцієнти моделей?
    \item Чи покращилася точність?
\end{itemize}

\subsection{Інтерпретація результатів}

\textbf{TODO:} Фізичне пояснення отриманих моделей:
\begin{itemize}
    \item Які ознаки виявилися найважливішими?
    \item Чи відповідає це фізичним очікуванням?
    \item Практична цінність отриманих моделей
\end{itemize}

\section{Обговорення}

\subsection{Переваги використання AI інструментів}

Використання AI інструментів для реалізації алгоритмів МГУА показало наступні переваги:

\begin{itemize}
    \item \textbf{Швидкість розробки:} початкова реалізація була створена за лічені години замість днів
    \item \textbf{Якість коду:} AI генерує структурований код з належними перевірками
    \item \textbf{Документація:} автоматичне додавання коментарів
    \item \textbf{Гнучкість:} легко модифікувати та додавати нові функції
    \item \textbf{Навчання:} процес роботи з AI допомагає краще зрозуміти алгоритм
\end{itemize}

\subsection{Виклики та обмеження}

\textbf{TODO:} Розширити опис викликів:
\begin{itemize}
    \item Необхідність точного формулювання завдань
    \item Важливість верифікації згенерованого коду
    \item Потреба в розумінні предметної області
    \item Ітеративний процес покращення
\end{itemize}

\subsection{Рекомендації щодо використання AI}

\textbf{TODO:} Додати практичні рекомендації:
\begin{itemize}
    \item Як формулювати завдання для AI
    \item Як перевіряти згенерований код
    \item Коли довіряти AI, а коли ні
    \item Як організувати ітеративний процес розробки
\end{itemize}

\section{Висновки}

\textbf{TODO:} Написати висновки, що підсумовують:

\begin{enumerate}
    \item Успішно застосовано сучасні AI інструменти (Claude Code) для реалізації комбінаторних алгоритмів МГУА на мові C
    \item Реалізовано два варіанти алгоритму:
    \begin{itemize}
        \item Квадратичні поліноми для пар змінних
        \item Лінійна багатовимірна регресія (класичний підхід)
    \end{itemize}
    \item Верифіковано коректність реалізації на тестовій задачі з академічної літератури:
    \begin{itemize}
        \item Алгоритм коректно ідентифікував ключові ознаки ($x_3, x_5, x_7, x_8$)
        \item Коефіцієнти моделей близькі до очікуваних
    \end{itemize}
    \item Застосовано розроблений інструмент до реальної задачі моделювання властивостей металевих сплавів
    \item Результати узгоджуються з попередніми дослідженнями [TODO: додати конкретику]
    \item Продемонстровано ефективність використання AI для наукового програмування
\end{enumerate}

\textbf{TODO:} Додати перспективи подальших досліджень

\section*{Подяки}

\textbf{TODO:} Додати подяки (якщо потрібно)

\begin{thebibliography}{99}

\bibitem{TODO1}
\textbf{TODO:} Додати посилання на оригінальні роботи Івахненка про МГУА

\bibitem{TODO2}
\textbf{TODO:} Додати посилання на статтю з тестовою задачею (example\_test\_sample.docx)

\bibitem{TODO3}
\textbf{TODO:} Додати посилання на попередню статтю про моделювання властивостей сплавів

\bibitem{TODO4}
\textbf{TODO:} Додати посилання на Claude Code / Anthropic

\bibitem{TODO5}
\textbf{TODO:} Додати інші релевантні посилання про AI coding tools

\end{thebibliography}

\end{document}
